\documentclass{scrartcl}

% These are only some keywords for the autocompletion-feature of many editors: section, subsection, subsubsection, paragraph,
% includegraphics, width, linewidth, linespread, figure, wrapfigure, caption, label, footnote, equation, input, cite, citetitle,
% citeauthor, footfullcite, tableofcontents, printbibliography, clearpage, frq, frqq, flq, flqq, grq, grqq, glq, glqq, textit,
% texttt, mathrm, dots, pmatrix, centering, phantom, minipage, ensuremath, hfill, vfill, 

\usepackage[
	hidelinks,
	breaklinks=true,
	pdftex,
	pdfauthor={Norman Koch},
	pdftitle={Emotional Algebra},
	pdfsubject={},
	pdfkeywords={},
	pdfproducer={},
	pdfcreator={}]{hyperref}

\delimitershortfall=-1pt

\newcommand{\centeredquote}[2]{
	\hbadness=5000
	\vspace{-1em}
	\begin{flushright}
		\item\frqq\textsl{#1}\flqq\ 
	\end{flushright}
	\nopagebreak
	\hfill ---\,\textsc{#2}\newline
	\vspace{-1em}
}

\newcommand{\centeredquoteunknownsource}[1]{
	\hbadness=5000
	\vspace{-1em}
	\begin{quotation}
		\begin{flushright}
			\item\frqq\textsl{#1}\flqq\ 
		\end{flushright}
	\end{quotation}
	\vspace{-1em}
}
\usepackage[utf8]{inputenc}
\usepackage[T1]{fontenc}
\usepackage[sc,osf]{mathpazo}
\usepackage{fourier}
\usepackage{soulutf8}
\usepackage{verbatim}
\usepackage{graphicx}
\usepackage{amsmath}
\usepackage{longtable}
\usepackage{amssymb}
\usepackage{babel}
\usepackage{lettrine}
\usepackage{xcolor}

\renewcommand{\LettrineFontHook}{\usefont{U}{yinit}{m}{n}}

\emergencystretch2em


\begin{document}

\author{Norman Koch}
\title{Emotional Algebra}

\maketitle

\tableofcontents
\newpage

\section{Preface}

\lettrine[nindent=0em]{\color{purple}T}{he Formalization}
of logic allowed us, a long period of time after it's discovery by Aristotle, to build computers that can outsource some of our
mental facilities. They allowed us to perform logical reasoning on a great number of propositions automatically.

We do very bad with things that cannot be put easily in front of us. A lot of problems humanity faces are the results of
denying subjective experience and declaring only physical events as true. Physical events are more easily formalized than
emotional ones and therefore offer a false feeling of more confidence in the results. Once something is formalized, we have
easier access to it because we can manipulate it in our mind without having to life through all of the real consequences.
Imagine an architect, building buildings which collapse every time until they don't anymore after `fixing all the bugs'.
This is unacceptable. But we do it in our emotional world all the time, because there is no easy way to externalizing them
and looking at their possible emotional consequences rigorously.

In this paper, I'd like to try to formalize emotions a similiar way we have formalized math. This, if it works, may 
lead to greater understanding of humans.

I will not try to explain what the basic words (like emotion) mean (\frq hypotheses non fingo\flq). You have to understand what they mean 
naturally, as you have to understand `sets' in set theory naturally. If you don't know what something like `feels bad' mean
think seriously about cutting off your arm. This would have consequences that you subjectively most likely consider
`bad' in the now (pain) and the future (inability to use arm).

\section{Conditions}

We need to define some conditions that affect human emotion first. I believe these are the most important facilities that
do that.

\subsection{Time}

Every human emotion is resided in the now. Even if we think about past or worry about the future, we do it \textit{now}.
And the now, although it always is, never really is. Once contemplated, it is already gone and slipped into the past.

Therefore, we need to define a time operator. I will suggest these:

\begin{tabular}{l|l}
	$E$           & $E$ is any emotional event \\
	$n E$ & $E$ depends mainly on the current situation \\
	$p E$ & $E$ depends mainly on a past situation \\
	$f E$ & $E$ depends mainly on future situations (`worry') \\
	$(n \wedge f) E$ & $E$ depends on the current situation now and the future \\
	$(f \wedge p) E$ & $E$ depends on the future and the past \\
	$(n \wedge p) E$ & $E$ depends on the now and the past \\
	$(n \wedge f \wedge p) E$ & $E$ depends on the now, the future and the past
\end{tabular}

\subsubsection{Now}

As every moment is always depending on the `now', we can treat it analogously to the multiplication operation 
and leave it out like we leave out the multiplication symbol if it should not be preciously stretched that the
situation is depending on the now and mainly the now, so $n E = E$. You can always set the n in front of
any equation since when you think about this equation, it is always `now', but I wil leave them out from here on
unless they are speficifically needed for emphasis.

\subsubsection{Past}

For this section, I will define reality as that on which external observers and I can agree to be the fact.

Sometimes, people misremember and therefore construct a past for themselves that did not really happen. In formalization
I suggest to label events with $\mathrm{rel}(t, E)$ if this happens, where $t$ is the level of accuracy of the
remembrance, where 1 means very reliable memory and 0 means very unreliable memory. Therefore, a complete sentence would be:

\begin{equation} p \left( \textrm{rel}(t, E_1) \wedge E_2 \right) \end{equation}

This would mean that the current situation is depending on a misremembered event $E_1$ and a real event
$E_2$.

\subsubsection{Future and uncertainty}

As the past is not here anymore and cannot be known with the same reliability than the present, the future
cannot be known with any reliability. Therefore, future events are doubtful and uncertain. 

Doubt is the human condition of not-being-sure. Though I deny the possibility of absolute reliability we
may define a formalization of doubt as $\textrm{rel}(t, E)$, where $t$ is between 0 and 1, where 0 means
`I know nothing about the possibility of this event' and 1 means `I feel certain that this event will happen'.

An example:

\begin{equation} f \left( \textrm{rel}(0.1, E_1) \wedge E_2 \right) \end{equation}

This means:

``My current emotional situation depends on two things that depend on something tomorrow, of which one ($E_1$) will likely not
occur and $E_2$ will certainly occur''

Leaving out the $\textrm{rel}(t, E)$ defaults to $t = 1$, meaning $E = \textrm{rel}(1, E)$.

\subsection{Emotional events}

Every emotional event can be ordered in roughly 3 categories. `Good', `bad' and 'neutral'. This is the way an honest
person would describe the emotion when asked how it felt. 

Examples.

Eating a really good meal in a good atmosphere results in an event that would be described as `good'. 

Running for your life would result in an event called `bad'.

Having a catatonic schizophrenic episode (the ones where people just don't react to anything and stop feeling entirely)
can be considered als neutral (since the person does not feel anything anymore).

I would like to define operators to describe these 3 categories. All of the arrow-operators are right-associative and
only get one parameter, which is why you can leave out brackets without losing information about which event is meant.

These operators are not exclusive, because every event can have facets that are good, bad and neutral altogether.

\begin{tabular}{l|l}
	$\uparrow E$ & Event $E$ is seen as good \\
	$\downarrow E$ & Event $E$ is seen as bad \\
	$\rightarrow E$ & Event $E$ is seen as emotionally neutral \\
	$\uparrow^{0.1} E$ & Event $E$ is a bit good ($\uparrow = \uparrow^1$) \\
	$\downarrow_{0.9} E$ & Event $E$ is a very bad ($\downarrow = \downarrow_1$) \\
	$\rightarrow^{1} E$ & Event $E$ is seen as emotionally neutral but I still care about it some way\footnotemark \\
	$\rightarrow_{0} E$ & Event $E$ is seen as emotionally neutral and I don't care about it\footnotemark\\
	$\updownarrow E$ & One is torn-apart if the event is good, bad or neutral (doubtful, uncertain), equal to $\uparrow E \wedge \downarrow E \wedge \rightarrow E$
\end{tabular}

\footnotetext{E.g. a situation where somebody tells me the number of rice corns eaten every year}
\footnotetext{E.\,g. the fact that Donald Knuth was born on the 10th of January 1938,
		which I care about, but it has no emotional significance over any other date that could be his birthday.}

Leaving out the arrow should be a way for shortening it from $\rightarrow E$, so that $E = \rightarrow E$.

\begin{equation} f \left( \textrm{rel}(0.1, \uparrow E_1) \wedge \downarrow E_2 \right) \end{equation}

``My current emotional situation depends on two things that depend on something tomorrow, of which one ($E_1$) will likely not
occur which I am very happy about and which is important to me and $E_2$ will certainly occur which I am negative about''

\subsection{Real Situations}

Every moment is a complex mixture of influences from the (real and perceived) past, present and future. But some
of these components are more important than others. The number of roof tiles in the room where I was born is
an event in the past, but has barely any meaning to my situation at all. The people my parents were on the other
hand we're very important. Therefore, it is important to give a formalism to ascribe importance. I suggest:

\begin{equation} \textrm{imp}(i, E) \end{equation}

That means event $E$ is $i$ important, where $i$ is between 0 (not important at all) to 1 (very important).

\begin{equation} 
	f \left(
		\textrm{imp}\left(
			1, \textrm{rel}\left(
				0.9, \uparrow E_1
			\right)
		\right)
		\wedge
		\left(
			\textrm{imp}\left(
				0.1, \downarrow E_2
			\right)
		\right)
	\right)
\end{equation}

``My current emotional situation depends on two things that depend on something tomorrow, of which one ($E_1$) will likely
occur which is very important to me which I am very happy about and $E_2$ will certainly occur which I would be a bit negative about
of which I also would don't really care much about if it happens''

An example situation would be that I am very happy that the sun will rise tomorrow again ($E_1$) (most probably) and I hope that the moon
does not crash into the earth before that ($E_2$), which I find highly unlikely to happen.

\subsection{Atmosphere}

Every situation is filled with an atmosphere. An atmosphere is the general way a world around one is perceived in the moment of perception.

Atmospheres consist of a mix of expectations what will happen in a place (future) and what has happened (past) without them
really neccessarily happening in the moment of perception. I will use the same Formalization as with Events,
$n A$ being an atmosphere which is what it is because of what is currently happening, $p A$ an atmosphere that
depends on the past and $f A$ an atmosphere that depends on the future.

Think of dark alley at night being scary or a swimming pool party with friends
in summer as happy, so, as with Events, they get arrows indicating good $\uparrow A$, bad $\downarrow A$ and neutral $\rightarrow A$.

A good atmosphere now ($n\uparrow A$) in a place which is based on good events that happened there in the past (p $\uparrow E$)  can be written as:

\begin{equation}n\uparrow A \supset \left(
			p \uparrow E
	\right)
\end{equation}

\subsection{Leib}

The Leib (old german for `body') is not the body. The Leib is where we feel ourselves. We don't feel the bodily actions of severing
nerves and so on, but pure pain when severing off a hand. The pain may be the result of a bodily event, but is not equal in extension,
as pain is something mental. 

Let's use $L$ for a `Leibliches' Event. 

There are many different ways that the Leib can interact with our emotions. We can feel free for example. Think about having achieved
a lot in a day and then going to bed happily, free of obligations left on that day. Or we can feel trapped, when we are in a situation
where we don't know how to get out. These Events $L$ can therefore be either good ($\uparrow$), bad ($\downarrow$) or neutral ($\rightarrow$),
which can also be supplemented with numbers like $\downarrow_1 L$ for `10 of 10'-pain or $\uparrow^1 L$ for a moment of great pleasure
(e.\,g. a well-earned orgasm with a person one loves). Leaving out the number set's it to 1.

Generally, these can be sorted like Events, as they are a special subcategory of Events. We have the modifiers $\uparrow$, $\downarrow$
and $\rightarrow$ to ascribe how we feel towards something. Cutting off ones arm would probably be $\downarrow L$, eating something
really delicious and healthy most probably leads to something like $\uparrow L$.

It's important to note that these must not be `objective' criteria, but more subjective ones. 

One may feel that cutting off ones arm ($H$)
may be a good action ($\uparrow$), even though it turns out not to be in the future  which makes us unhappy in the future ($\downarrow f E$).
This would be formalized as:

\begin{equation}
	f \downarrow \left( f\downarrow E \supset \uparrow H \right)
\end{equation}

\subsection{Actions}

Sometimes we act like we do because of the past, sometimes because of the present and sometimes because of the future, and sometimes
a complex mix of these. For example, you might have been screwed in a contract earlier (past), and now you set up a better contract
for not getting screwed in the future. 

Let's call Acts $H$ for german `Handlung' (Act).

Sometimes acts don't turn out the way you wanted them to go. For acts that turn out good, let's use $\uparrow H$, for bad ones 
$\downarrow H$, for neutral ones $\rightarrow H$ and for mixtures of these $\updownarrow H$. As with events and atmospheres,
let's assign values between 0 and 1, like: $\uparrow^{0.9} H$ for an act that turned out to be good or $\downarrow_1$ for an
act that turned out to be very bad.

An act can follow situations and atmospheres. Think of running away ($H$) from the dark alley mentioned earlier. You do this because the
atmosphere $A$ is bad because you know several people were murdered there ($E$) in the past and you want to go away from there right now. This would be
described as:

\begin{equation}
	n \left(\left(
		\downarrow A \supset p: \downarrow E
	\right) \supset H\right)
\end{equation}

\subsection{Motivations}

Let's call Motivations $M$. Not every act we do is fully motivated by conscious reasons. 

If important, we can write this with $\textrm{cons}(c, M)$, where $c$ is between 0 and 1, where
0 means `fully unconscious' and 1 means 'fully conscious'.

When going into `autopilot', e.\,g. in a dangerous situation, $c$ will be very low. Imagine again running
away from war. The Motivation $M$ is to survive, but it is so basic that you don't need much conscious reflection
on that.

Running away ($H$) successfully ($\uparrow$) from war ($\downarrow E$) to survive ($M$) requires lot consciousness ($\textrm{cons}(0, M)$) would be formalized as:

\begin{equation} 
	\uparrow H \supset\left(
		\left(n\wedge p\wedge f\right)E \supset \textrm{cons}(0, M)
	\right)
\end{equation}

Successfully and very consciously moving your body ($H$) to get to a party ($E$) in the future you are happy about ($\uparrow$) can be written as:

\begin{equation}
	\left(
		\uparrow f E \supset \textrm{cons}(1, M)
	\right)
	\supset \uparrow H 
\end{equation}

\subsection{Situations}

\subsection{Outline}

There are these types of things that control emotion:

\begin{tabular}{l|l}
	$x$ & Type \\ \hline
	$E$ & Events \\
	$M$ & Motivations \\
	$H$ & Actions \\
	$A$ & Atmospheres \\
	$L$ & Leib
\end{tabular}

All of them must be described together to form a complete description of a situation. 
The events that lead to it and how they were perceived ($p m E$), the motivations that lead to these events ($p m M$)
and control the current acts ($p m H$), the atmospheres in the place ($p m A$) and the Leib and for each of them information
about the reliability- (rel), importance- (imp) and consciousness- (cons) levels.

To form a Situation S, you need an equation in the form of ($\{\}$ being sets of those data):

\begin{equation}
	S :=: 
\end{equation}

\begin{equation}
	\left.
		\mathrm{imp}\left(E_i, \left\{\textrm{rel}\left(r_{E_1}, E_1\right) \wedge \textrm{imp}\left(i_{E_1}, E_1\right) \wedge \dots\right\}\right) \cup
	\right.
\end{equation}

\begin{equation}
	\left.
		\mathrm{imp}\left(M_i, \left\{\textrm{rel}\left(r_{M_1}, M_1\right) \wedge \textrm{imp}\left(i_{M_1}, M_1\right) \wedge \dots\right\}\right) \cup
	\right.
\end{equation}

\begin{equation}
	\left.
		\mathrm{imp}\left(H_i, \left\{\textrm{rel}\left(r_{H_1}, H_1\right) \wedge \textrm{imp}\left(i_{H_1}, H_1\right) \wedge \dots\right\}\right) \cup
	\right.
\end{equation}

\begin{equation}
	\left.
		\mathrm{imp}\left(A_i, \left\{\textrm{rel}\left(r_{A_1}, A_1\right) \wedge \textrm{imp}\left(i_{A_1}, A_1\right) \wedge \dots\right\}\right) \cup
	\right.
\end{equation}

\begin{equation}
	\left.
		\mathrm{imp}\left(L_i, \left\{\textrm{rel}\left(r_{L_1}, L_1\right) \wedge \textrm{imp}\left(i_{L_1}, L_1\right) \wedge \dots\right\}\right)
	\right.
\end{equation}

Each of these can be prepended with one or more of the operators $\uparrow$ (good/successful), $\downarrow$ (bad/unsuccessful) and $\rightarrow$ (neutral).

Each of these modifiers can have options from 0 to one, where positive options are in superscript and negative actions are in subscript, like
very good actions $\uparrow^1$ and very bad actions $\downarrow_1$. When no number is written, it will default to 1, so that $\uparrow = \uparrow^1$.

Each of these sets has a imp (importance) function with a specific parameter. The set of these parameters, i.\,e. $\{E_i, M_i, H_i, A_i, L_i\}$, define the temperament of the
person. Some people's temperament is more predicated on inner motivation $M$ than on outter or inner events $E$, so for these, the importance values are $M_i > E_i$.

These value-sets can be simplified like this:

\begin{equation}
	\mathrm{imp}\left(L_i, \left\{\textrm{rel}\left(r_{L_1}, L_1\right) \wedge \textrm{imp}\left(i_{L_1}, L_1\right) \wedge \dots\right\}\right) =
	\left\{\mathrm{imp}\left(L_i, \textrm{rel}\left(r_{L_1}, L_1\right)\right) \wedge \textrm{imp}\left(i_{L_1} \cdot L_i, L_1\right) \wedge \dots\right\}
\end{equation}

% The sets internally get defined by cartesian products % TODO

Any of these can lay in the past, present or future or any combinations of these, but at least one. When nothing is written, $x = p x$. 

Every emotion is always bound in time and felt right now, therefore, every expression must start with

\begin{equation} n x \end{equation}

where $x$ is any of $E, M, H, A$ or $L$. Therefore, the `$n$' can be left away, meaning $x = n x$. Also, any of these can
be dependent on the past $p$, present ($n$) and/or future ($f$).

An Event $E$ can consist of multiple sub-factors, like having gone to a party successfully ($p \uparrow H$) and now experiencing a good atmosphere ($n \uparrow A$) 
there would be formalized as:

\begin{equation}
	\left(
		p \uparrow H \wedge n \uparrow A
	\right)
	\supset n \uparrow E 
\end{equation}

Some good emotional events $\uparrow E$ that happens in the future because of something you do now would but don't really want to do (like cleaning up
your room) ($\downarrow A$) would be formalized as:

\begin{equation}
	n \downarrow A \supset f \uparrow E
\end{equation}

Some of these values have specification-functions.

Events $E$ can me misremembered or remembered correctly. They can be modified with $\textrm{rel}(t, E)$, where $t = 0$ for very unreliable and
$t = 1$ for very reliable subjective memories of the events. For future events, This can be used as a reliability-meter how certain (0 = not all certain
and 1 = very certain) we are that they will occur. We can define that $M = \textrm{rel}(1, E)$.

Motivations $M$ can be modified with the level of consciousness they require with $\textrm{cons}(t, M)$, where $t = 0$ means zero consciousness required
(e.\,g. the will to survive) and $t = 1$ meaning high level of consciousness required, e.\,g. solving mathematical equations in order to obtain a goal.
Leaving this out will mean $\textrm{cons}(1, M) = M$.

All of these situations $x$ can be ascribed with an importance level, where $t = 0$ no importance and $t = 1$ is extreme importance. We define this as
$\textrm{imp}(t, x)$. When left out, we define $x = \textrm{imp}(1, x)$.

\section{Symbols and parameters}

\begin{tabular}{l|l}
	Symbol & Type \\ \hline
	$I$ & Any item possible or combination of items \\
	$p, f, n$ & Past, future and present \\
	$\uparrow, \downarrow, \rightarrow$ & Positive, negative and neutral \\
	$(, )$ & Brackets, create groups of elements in order to overcome standard precedence rules \\
	$\wedge$ & Logical and, $I_1 \wedge I_2 \wedge I_3 =  I_1 \wedge (I_2 \wedge I_3) = (I_1 \wedge I_2) \wedge I_3$\\
	$I_1 \supset I_2$ & $I_2$ depends on $I_1$, $I_1 \supset I_2 \supset I_3 = I_1 \supset (I_2 \supset I_3)$\\
	$I_1 \lor I_2$ & Locigal or, $I_1 \lor I_2$ \\
	$\lnot I_1$ & Negates $I_1$, so that $\lnot \downarrow I_1 = \uparrow I_2$ and $\lnot \textrm{cons}(1, I_1) = \textrm{cons}(0, I_1)$ and $(\lnot E)$ meaning `$E$ does not occur'\footnotemark
\end{tabular}

\footnotetext{Here grouping with brackets is important, since $\lnot \uparrow E$ may mean either `the god event $E$ will not occur' or $\downarrow E$, but with $(\lnot \uparrow) E$ it is clear
that what is meant is `the event $E$ is bad' ($\downarrow E$).}

\section{Inference Rules}

The modifier functions $\textrm{imp}$ (importance), $\textrm{rel}$ (reliability) and $\textrm{cons}$ (consciousness required) are defined like this:

\begin{equation} \textrm{imp}(t_1, E_1) \wedge \textrm{imp}(t_2, E_2) :=: \textrm{imp}(\frac{t_1 + t_2}{2}, E_1 \wedge E_2) \end{equation}

So having to tasks $E_1, E_2$ that are 0.8 and 0.7 important respectively can be described as

\begin{equation} \textrm{imp}(0.8, E_1) \wedge \textrm{imp}(0.7, E_2) = \textrm{imp}\left(\frac{0.8 + 0.7}{2}, E_1 \wedge E_2\right) = \textrm{imp}(0.75, E_1 \wedge E_2)\end{equation}

Having them or'ed multiplies the values:

\begin{equation} \textrm{imp}(0.8, E_1) \lor \textrm{imp}(0.7, E_2) = \textrm{imp}\left(0.8 \cdot 0.7, E_1 \lor E_2\right) = \textrm{imp}(0.56, E_1 \lor E_2)\end{equation}

As a single event can have both negative and positive consequences simultaneously, we define

\begin{equation}n \updownarrow E :=: f \downarrow E \wedge p \uparrow E\end{equation}

Positive and negative \textit{do not} add up to neutral, but rather to torn-apart ($\updownarrow$):

\begin{equation}\updownarrow E :=: \downarrow E \wedge \uparrow E\end{equation}

\begin{equation}\updownarrow \left(E_1 \wedge E_2\right) :=: \left(\downarrow E_1\right) \wedge \left(\uparrow E_2\right) \end{equation}

Neutral events do not change the output of an equation, like:

\begin{equation}\downarrow E :=: \downarrow E_1 \wedge \rightarrow E_2\end{equation}

The event assesment-symbols can be pulled from several single events like this:

\begin{equation}\downarrow E_1 \wedge \downarrow E_2 :=: \downarrow \left(E_1 \wedge E_2\right)\end{equation}

\begin{comment}
\section{Example}

Let's construct the XOR-Operator $\oplus$. It is true for $a, b$ iff $(a \wedge \lnot b) \lor (\lnot a \wedge b)$ .

Let $a, b$ be any type of Events consisting of a time string $t$, a positive-negative-neutral-modifier $m$ and the Events $a = E_1, b = E_2$.

We have an equation of the form:

\begin{equation}
	\left(
		\left(
			t_1 m_1 E_1 
		\right)
		\lor
		\lnot \left(
			t_2 m_2 E_2
		\right)
	\right)
	\lor 
	\left(
		\lnot\left(
			t_1 m_1 E_1
		\right) 
		\lor 
		\left(
			t_2 m_2 E_2
		\right)
	\right)
\end{equation}

\end{comment}


\end{document}
